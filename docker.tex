\documentclass[12pt]{article}
\usepackage[utf8]{inputenc}
\usepackage{amsmath}
\usepackage{graphicx}
\usepackage{hyperref}
\usepackage{listings}
\usepackage{xcolor}

\title{Docker Crash Course Summary}
\author{Carlos Padilla}
\date{July 2024}

\begin{document}

\maketitle

\begin{abstract}
This document provides a comprehensive summary of the main concepts of Docker, including hands-on experience, installation, Docker images, containers, registries, Dockerfile, and more. The information is designed to be extensive and detailed, making it ideal for individuals looking to level up their engineering skills with Docker.
\end{abstract}

\section{Introduction}

\textbf{In this video, I will teach you all the main concepts of Docker including getting your first hands-on experience with it.} If you have to use Docker at work or if you need to learn Docker to level up your engineering skills and need to get started fast and understand all the main concepts and learn basics of how to work with Docker, this crash course is exactly right for you.

\section{What is Docker?}
Docker is a virtualization software that makes developing and deploying applications very easy, much easier compared to how it was done before Docker was introduced. Docker does this by packaging an application into something called a \textbf{container} that has everything the application needs to run: the application code, its libraries and dependencies, the runtime, and the environment configuration.

\subsection{Why Docker?}
Docker was created to solve several problems in engineering and software development. Before Docker, developers had to install and configure all the services an application depended on directly on their operating system. This process was error-prone and tedious, especially in teams where different developers used different operating systems.

\section{Installing Docker}
To install Docker, you simply go to the official Docker website and follow the installation guide. Docker Desktop is available for Windows, Mac, and Linux. It includes everything you need to get started with Docker, including the Docker engine, a command-line interface (CLI) client, and a graphical user interface (GUI) client.

\section{Docker Images and Containers}
\textbf{Docker images} are packages that contain everything needed to run an application. They include the application code, runtime, libraries, environment variables, and configuration files. \textbf{Containers} are running instances of Docker images.

\subsection{Working with Docker Images}
To download an image from Docker Hub, you use the \texttt{docker pull} command. For example, to download the Nginx image, you would use:
\begin{lstlisting}
docker pull nginx:latest
\end{lstlisting}

\subsection{Running Containers}
To run a container from an image, you use the \texttt{docker run} command. For example, to run the Nginx image as a container, you would use:
\begin{lstlisting}
docker run -d -p 8080:80 nginx:latest
\end{lstlisting}
This command runs the Nginx container in detached mode (\texttt{-d}) and maps port 80 in the container to port 8080 on the host (\texttt{-p 8080:80}).

\section{Creating Your Own Docker Images}
To create your own Docker image, you need to write a Dockerfile. A Dockerfile is a text document that contains all the commands to assemble an image. Here is an example Dockerfile for a simple Node.js application:
\begin{lstlisting}
# Use the official Node.js image as the base image
FROM node:14

# Create and set the working directory
WORKDIR /app

# Copy package.json and install dependencies
COPY package.json ./
RUN npm install

# Copy the rest of the application code
COPY . .

# Expose the port the app runs on
EXPOSE 3000

# Start the application
CMD ["node", "server.js"]
\end{lstlisting}

\section{Docker Registries}
Docker images are stored in registries. Docker Hub is the largest public registry, but there are also private registries available from cloud providers like AWS, Google Cloud, and Azure.

\section{Advanced Docker Concepts}
\subsection{Docker Compose}
Docker Compose is a tool for defining and running multi-container Docker applications. With Docker Compose, you use a YAML file to configure your application's services, and then you create and start all the services with a single command.

\subsection{Docker Volumes}
Docker volumes are used to persist data generated by and used by Docker containers. They are the preferred mechanism for persisting data because they are managed by Docker and can be shared between containers.

\section{Conclusion}
By the end of this crash course, you should feel more confident in your knowledge and understanding of Docker. You can build on this foundational knowledge to become a Docker power user, mastering advanced topics such as Docker Compose, Docker volumes, and integrating Docker into CI/CD pipelines.

For further learning, check out the resources provided under the video description. Be sure to subscribe to our channel for more tutorials and connect with us on social media for behind-the-scenes content and weekly updates.

\section{Additional Resources}
For more advanced Docker topics and to practice your skills further, explore the following:
\begin{itemize}
    \item Docker official documentation: \url{https://docs.docker.com/}
    \item Docker Compose: \url{https://docs.docker.com/compose/}
    \item Docker volumes: \url{https://docs.docker.com/storage/volumes/}
    \item DevOps bootcamps: Check out online platforms like Udacity, Coursera, and Pluralsight.
\end{itemize}

\end{document}

